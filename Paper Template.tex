% LaTEX source code
% Last modified November 1st, 2005
% Steve Miller

% note that the percent sign comments out the rest of the line
% first, we set a document class. often use 12pt characters, though
% sometimes people do 11 or 10. you can do report or article, both similar

%\documentclass[12pt,letterpaper]{article}
\documentclass[12pt,reqno]{amsart}


\addtolength{\textwidth}{2cm} \addtolength{\hoffset}{-1cm}
\addtolength{\marginparwidth}{-1cm} \addtolength{\textheight}{2cm}
\addtolength{\voffset}{-1cm}


% below are some packages that are needed for certain symbols, graphics, colors.
% safest to just include these.

\usepackage{times}
\usepackage[T1]{fontenc}
\usepackage{mathrsfs}
\usepackage{latexsym}
\usepackage[dvips]{graphics}
\usepackage{epsfig}
%\usepackage{hyperref, amsmath, amsthm, amsfonts, amscd, flafter,epsf}
\usepackage{amsmath,amsfonts,amsthm,amssymb,amscd}
\input amssym.def
\input amssym.tex
\usepackage{color}


    %=======================================================
    %   THIS IS WHERE YOU PUT SHORTCUT DEFINITIONS
    %========================================================

% Note that we use a percent sign to comment out a line
% below are shortcut commands

%%%%%%%%%%%%%%%%%%%%%%%%%%%%%%%%%%%%%%%%%%%%%%%
% below are shortcuts for equation, eqnarray,
% itemize and enumerate environments

\newcommand\be{\begin{equation}}
\newcommand\ee{\end{equation}}
\newcommand\bea{\begin{eqnarray}}
\newcommand\eea{\end{eqnarray}}
\newcommand\bi{\begin{itemize}}
\newcommand\ei{\end{itemize}}
\newcommand\ben{\begin{enumerate}}
\newcommand\een{\end{enumerate}}

%%%%%%%%%%%%%%%%%%%%%%%%%%%%%%%%%%%%%%%%%%%%%%%%
% Theorem / Lemmas et cetera

\newtheorem{thm}{Theorem}[section]
\newtheorem{conj}[thm]{Conjecture}
\newtheorem{cor}[thm]{Corollary}
\newtheorem{lem}[thm]{Lemma}
\newtheorem{prop}[thm]{Proposition}
\newtheorem{exa}[thm]{Example}
\newtheorem{defi}[thm]{Definition}
\newtheorem{exe}[thm]{Exercise}
\newtheorem{rek}[thm]{Remark}
\newtheorem{que}[thm]{Question}
\newtheorem{prob}[thm]{Problem}
\newtheorem{cla}[thm]{Claim}


%%%%%%%%%%%%%%%%%%%%%%%%%%%%%%%%%%%%%%%%%
% shortcuts to environments
% this allows you to do textboldface: simply type \tbf{what you want in bold}

\newcommand{\tbf}[1]{\textbf{#1}}
\newcommand{\arr}[1]{\overrightarrow{#1}}
\newcommand{\avec}[1]{\langle #1 \rangle}

%%%%%%%%%%%%%%%%%%%%%%%%%%%%%%%%%%%%%%%%%%%%%%%%%%
% shortcut to twocase and threecase definitions

\newcommand{\twocase}[5]{#1 \begin{cases} #2 & \text{#3}\\ #4
&\text{#5} \end{cases}   }
\newcommand{\threecase}[7]{#1 \begin{cases} #2 &
\text{#3}\\ #4 &\text{#5}\\ #6 &\text{#7} \end{cases}   }


%%%%%%%%%%%%%%%%%%%%%%%%%%%%%%%%%%%%%%%%%
%Blackboard Letters

\newcommand{\R}{\ensuremath{\mathbb{R}}}
\newcommand{\C}{\ensuremath{\mathbb{C}}}
\newcommand{\Z}{\ensuremath{\mathbb{Z}}}
\newcommand{\Q}{\mathbb{Q}}
\newcommand{\N}{\mathbb{N}}
\newcommand{\F}{\mathbb{F}}
\newcommand{\W}{\mathbb{W}}
\newcommand{\Qoft}{\mathbb{Q}(t)}  %use in linux


%%%%%%%%%%%%%%%%%%%%%%%%%%%%%%%%%%%%%%%%%
% Finite Fields and Groups

\newcommand{\Fp}{ \F_p }


%%%%%%%%%%%%%%%%%%%%%%%%%%%%%%%%%%%%%%%%%
% Fractions

\newcommand{\foh}{\frac{1}{2}}  %onehalf
\newcommand{\fot}{\frac{1}{3}}
\newcommand{\fof}{\frac{1}{4}}

%%%%%%%%%%%%%%%%%%%%%%%%%%%%%%%%%%%%%%%%%
% Legendre Symbols

\newcommand{\js}[1]{ { \underline{#1} \choose p} }


%%%%%%%%%%%%%%%%%%%%%%%%%%%%%%%%%%%%%%%%%
% matrix shortcuts

\newcommand{\mattwo}[4]
{\left(\begin{array}{cc}
                        #1  & #2   \\
                        #3 &  #4
                          \end{array}\right) }

\newcommand{\matthree}[9]
{\left(\begin{array}{ccc}
                        #1  & #2 & #3  \\
                        #4 &  #5 & #6 \\
                        #7 &  #8 & #9
                          \end{array}\right) }

\newcommand{\dettwo}[4]
{\left|\begin{array}{cc}
                        #1  & #2   \\
                        #3 &  #4
                          \end{array}\right| }

\newcommand{\detthree}[9]
{\left|\begin{array}{ccc}
                        #1  & #2 & #3  \\
                        #4 &  #5 & #6 \\
                        #7 &  #8 & #9
                          \end{array}\right| }


%%%%%%%%%%%%%%%%%%%%%%%%%%%%%%%%%%%%%%%%%
% greek letter shortcuts

\newcommand{\ga}{\alpha}                  %gives you a greek alpha
\newcommand{\gb}{\beta}
\newcommand{\gep}{\epsilon}


%%%%%%%%%%%%%%%%%%%%%%%%%%%%%%%%%%%%%%%%%
% general functions
\newcommand{\notdiv}{\nmid}               % gives the not divide symbol


%%%%%%%%%%%%%%%%%%%%%%%%%%%%%%%%%%%%%%%%%%%
% the following makes the numbering start with 1 in each section;
% if you want the equations numbered 1 to N (without caring about
% what section you are in, comment out the following line.
\numberwithin{equation}{section}


\begin{document}

\title{Title of the Paper}

\author{Steven J. Miller}
\email{Steven.J.Miller@williams.edu} \address{Department of
  Mathematics and Statistics, Williams College, Williamstown, MA 01267}

%\author{Steven J. Miller\thanks{E-mail: \texttt{sjmiller@math.brown.edu}}}

\subjclass[2000]{11M06 (primary), 12K02 (secondary).}

\keywords{How to use TeX}

\date{\today}

\thanks{The author would like to thank to Daneel Olivaw for comments on an earlier draft. The author was supported by a grant from the University of Trantor, and it is a pleasure to thank them for their generosity.}

\begin{abstract}
The point of these notes is to give a quick introduction to some
of the standard commands of LaTeX; for more information see any
reference book. Thus we concentrate on a few key things that will
allow you to handle most situations. For the most part, we have
tried to have the text describe the commands; though of course we
cannot do this everywhere. You should view both the .tex code and
the output (either .pdf or .dvi) simultaneously.
\end{abstract}

\maketitle

\tableofcontents

\section{First Steps}

LaTeX is a cross between a word processor and a programming
language. The purpose is to allow one to write articles with lots
of mathematical symbols and equations easily. The default is to be
in non-mathematical mode; we discuss how to enter mathematical
mode below. The first few lines of the code describe the
formatting (font size is 12 for this, this is an article not a
report or book, which packages we need (some are needed for
including graphics, some for standard definitions of common
symbols or expressions), short cut commands, how we want theorems,
lemmas and the like to look, and so on). For now, you can keep
these settings and just modify the text below.

\subsection{General Mathematics}

Garbage text for format purposes. Here is math mode:
$\alpha^{\Gamma_3} + \beta^{12}_\gamma$. To enter math mode inside
text, simply type a dollar sign. Type another dollar sign to exit
math mode. Many of the symbols for everyday mathematics is as you
would expect: you start with a slash and then end with the
command. Thus lowercase Greek letters are slash letter, such as
slash alpha, slash beta, slash gamma: $\alpha$, $\beta$, $\gamma$.
Uppercase Greek letters (when they exist) are slash Gamma, slash
Delta and so on: $\Gamma$, $\Delta$. Sums, products and integrals
are slash sum, slash prod and slash int: $\sum$, $\prod$, $\int$.
Less than and equal to is slash le: $\le$; similarly greater than
or equal to is slash ge: $\ge$. Being an element of is slash in:
$\in$; not an element of is slash not slash in: $\not\in$. Subset
is slash subset $\subset$, and so forth. Note many of these are
exactly as you'd expect: $a \in A \subset B$ and $c \not\in B$.
Not equal to is slash neq and not divide is slash neq |: $4 \neq
5$ and $4 \nmid 5$.

Other common symbols like infinity are slash infty: $\infty$, the
section sign is slash S: $\S$, union is slash cup: $\cup$, and
intersection is slash cap: $\cap$. To do math blackboard is slash
mathbb \{ letter \}; thus the integers and rationals are
$\mathbb{Z}$ and $\mathbb{Q}$. We can put arrows over things by
slash overrightarrow \{ text \}: $\overrightarrow{v}$. For the
other direction it is overleftarrow, to underline it is overline,
for a hat (like the Fourier transform) it is widehat:
$\widehat{f}$. To have a square root it is slash sqrt \{ text \}:
$\sqrt{4x+1}$; if we want an nth root it is slash sqrt [n] \{ text
\}: $\sqrt[n]{4x+1}$.

For multivariable calculus, we could write $\overrightarrow{v}$ for the vector $v$, or we could use our shortcut command $\arr{v}$. If we want a cross product, we could write $\arr{v} \times \arr{w}$. If we want a specific vector, we can do $\avec{1,2,3,4,5}$.

Certain symbols (dollar signs, percent sign, pound sign, ambersand
sign) are used in tex to mean other things: if we want these in
texts we put a slash before them: \$, \%, \#, \&, \{, \},
$\backslash$. The slash backslash gives a back slash. For example,
the percent sign is used for comments that you do not want the tex
compiler to read; anything on the line with the
percent is commented out and not displayed. % thus you cannot see this.

Many standard functions have shortcuts to make them look good: for
example, $\backslash$cos, $\backslash$sin, $\backslash$tan,
$\backslash$log and so on: $\cos$, $\sin$, $\tan$, $\log$. Compare
how things look with and without the $\backslash$: $\cos(\log(x) +
1)$ versus $cos(log(x)+1$; without the $\backslash$it italicizes.
In fact, in math mode any text is automatically italiziced. We
discuss later how to handle text in math mode.

To do superscripts use a carot (shift-six) and to do subscripts
use underscore (shift-minus sign). Thus x-squared is $x^2$ and
integrating f(x) from a to b is $\int_a^b f(x)dx$. If you have
more than one character as the superscript or subscript, you  must
surround it by left and \}s: compare $x^2_{i+j}$ to $x^2_i+j$.
Sometimes, to be safe, one surrounds both subscripts and
superscripts with braces.

If there are frequently used expressions, it is worthwhile to
define shortcuts. Instead of typing $\backslash$alpha to get an
alpha, I can type $\backslash$ga, because I have defined that to
be a Greek alpha: $\ga$. Similarly for not divide it is
$\backslash$notdiv (as a shortcut for $\backslash$nmid): $4
\notdiv 5$ versus $4 \nmid 5$. The way we do shortcuts is earlier
in the document, we do $\backslash$newcommand \{
$\backslash$shortcut name \} \{ command \}. Thus for example the
shortcut for greek lowercase alpha is $\backslash$newcommand \{
$\backslash$ga \} \{ $\backslash$alpha \}.

To do fractions, type $\backslash$frac \{ numerator \} \{
denominator \}: $\frac{12}{34}$. If you have the ratio of two
one-digit numbers, you do not need the braces: $\frac12$; however,
compare $\frac{1}{12}$ to $\frac112$. These are common errors in
tex (forgetting braces; the same also happens often with exponents
or subscripts); that is why often it is best to include the
braces. Remember every \{ must be paired with a \}.

There are many different text enviroments; for webpages a good one
to do is slash texttt \{insert text here \}, which would appear as
$$\texttt{http://www.math.brown.edu/~sjmiller}.$$ Notice when we do
this that the tilde is not displayed; to have it display use \$slash
sim\$: thus we would write
$$\texttt{http://www.math.brown.edu/$\sim$sjmiller}.$$

Some people might have latex environments where it will replace a
$\alpha$ with the Greek letter alpha. Note that if you type
$\alpha$ it might replace it with what a real alpha, while if I
type $\ga$, the user defined shortcut, it does not replace.


\subsection{One Line Equations}

Here's how to do an equation. Once you type $\backslash$begin \{
equation \}, you've automatically entered math mode. Now if you
type anything, say $\backslash$Lambda or $\backslash$epsilon or
even some of our user-defined shortcuts, they will be properly
formatted:

\begin{equation}
\frac{\Lambda'(s)}{\Lambda(s)} = s + \epsilon \int_a^b 3x^2
e^{2\pi i x} dx.
\end{equation}

In the above, frac (with a $\backslash$before it) gives a
fraction; it puts the first thing in curly brackets as the
numerator, and the second as the denominator. To do Greek letters,
type $\backslash$letter (for example, $\backslash$epsilon), while
an integral is $\backslash$int, a sum is $\backslash$sum, and so
on.

I've made a shortcut for equations: $\backslash$be for begin
equation and $\backslash$ee for end equation. It looks as follows

\be \frac{\Lambda'(s)}{\Lambda(s)} = \epsilon \int_a^b 3x^2
e^{2\pi i x} dx. \ee

If you have a blank line between the text and the equation, it
sometimes inserts space, and indents the next text as it is a new
paragraph; if there is no blank line, it sometimes keeps them
closer, and does not indent the following text.

For example, let us compare the following to the earlier examples
(where we had blank lines). Thus we follow this text with
\begin{equation} \frac{\Lambda'(s)}{\Lambda(s)}\ =\ s + \epsilon
\int_a^b 3x^2 e^{2\pi i x} dx.
\end{equation} As there is no carriage return, this text is part
of the same paragraph, and is thus not indented.

As long as we are discussing equations, let us consider the case
when we have more involved expressions, for example, we might have
\begin{equation} (\frac{\sin(x^2+3x)}{x}+\frac{4}{e^x})^2.
\end{equation} Note how the left and right parentheses are too
short. There are two ways to fix this. One way is to use
$\backslash$Big ( and $\backslash$Big ) (and Bigg if you need a
bigger parentheses). A better way is to use $\backslash$left ( and
$\backslash$right ); this automatically adjusts the size of the
parentheses. Unlike the Big or Bigg commands, if you have a
$\backslash$left ( you must match it with a $\backslash$right ).
Thus
\begin{equation}
\left(\frac{\sin(x^2+3x)}{x}+\frac{4}{e^x}\right)^2.
\end{equation}

\subsection{Labeling Equations}\label{sec:labellingeqs}

What if I want to keep track of the equation number, so that I can
refer to it in the text? For example, consider
\begin{equation}\label{eqlambdaprimelambda}
\frac{\Lambda'(s)}{\Lambda(s)} = \epsilon \int_a^b 3x^2 e^{2\pi i
x} dx.
\end{equation} The way we label an equation is, after the
$\backslash$begin \{ equation \}, we
write $\backslash$label \{ labelname \}.

Note the label. I can choose anything (as long as there are no
numbers, just letters) for the name. I choose to label all
equations with eq followed by a descriptive name; lemmas I start
lem followed by a descriptive name, and so on.

To refer to the equation, I merely have to write $\backslash$ref
\{ eqname \}; for example for the previous equation it would be
\ref{eqlambdaprimelambda}; however, it's better to write Equation
\ref{eqlambdaprimelambda} or \eqref{eqlambdaprimelambda}. The
difference between $\backslash$ref and $\backslash$eqref is that
$\backslash$eqref automatically puts parentheses around the
equation number.

Whenever you add equations, you have to compile Latex twice to get
the references correct. The advantage of using equation labels is
that, if we add additional equations before the equation we want
to refer to, the equation number of our equation changes; by using
labels these are automatically updated.

For example, consider the following.
\begin{equation}\label{eq:gf1}
g(x) = \int_0^\infty f(x,y) dy.
\end{equation} We can refer to this and say by \eqref{eq:gf1} of
\S\ref{sec:labellingeqs} we have.... You can label equations (or
sections, theorems and so on) any way you want, though it is nice
to have a standard method. I prefer to label equations by eq colon
name, theorems by thm colon name, sections by sec colon name,
sub-sections by subsec colon name, and so on.


\subsection{Multi-Line Equations: Eqnarray}

What if your equation is more than one line? You might want to use
eqnarray instead of equation. The $\backslash$nonumber
$\backslash$$\backslash$ is a carriage return without numbering
that line; personally, I like to wait to the last line to number
something. Here's an example:
\begin{eqnarray}
\frac{\Lambda'(t)}{\Lambda(t+1)} & = & f(x) g(s) + f(x-t) -
g(s)f(x) \nonumber\\ \frac{\Lambda'(t)}{\Lambda(t+1)} & = & f(x -
t).
\end{eqnarray}

Again, if I don't want to type begin eqnarray I can use the
shortcut (that I defined above): \bea
\frac{\Lambda'(t)}{\Lambda(t+1)} & = & f(x)
g(s) + f(x-t) - g(s)f(x) \nonumber\\
\frac{\Lambda'(t)}{\Lambda(t+1)} & = & f(x - t). \eea

Here, I've chosen to use bea to stand for begin equation array.
You can define your shortcuts almost freely (you can't use numbers
in a shortcut definition).

The formatting is done by the ampersand signs, \&. (Note: if you
have a special symbol which you want to display in Latex, you put
a $\backslash$before it. Thus, to print a percent-sign in math
mode is \%, or to print a pound sign is \#.) The eqnarray
environment has two ampersands per line, and centers the lines on
what is between the ampersands.

Usually, one does not repeat the left hand side. Thus, it is more
natural to write

\begin{eqnarray}
\frac{\Lambda'(t)}{\Lambda(t+1)} & = & f(x) g(s) + f(x-t) -
g(s)f(x) \nonumber\\  & = & f(x - t).
\end{eqnarray}

Here's a somewhat lengthier example:

\begin{eqnarray}
\frac{1}{m}\sum_p^{m^\sigma} p^{-\foh} & \le & \frac{1}{m} (
\sum_p^{m^\sigma} \frac{1}{p})^{\foh} ( \sum_p^{m^\sigma}
1)^{\foh} \nonumber\\ & \le & \frac{1}{m} (\log \log m^{\sigma} +
A)^{\foh} (Li(x) + O(x^{\foh} \log x))^{\foh} \nonumber\\ & \ll &
\frac{1}{m} (\log \log m)^{\foh} (\frac{2m^{\sigma}}{\log
m})^{\foh} \nonumber\\ & \ll & m^{\foh \sigma - 1} (\frac{\log
\log m}{\log m})^{\foh}.
\end{eqnarray}

In the above, I have used a user defined command, $\backslash$foh.
That is a shortcut I've created to write $\frac{1}{2}$. If there
is something you use many times, you should have a shortcut for
it.

Note several things look wrong in the above equation. The
parentheses are wrong-sized, and the mathematical function Li (the
logarithmic integral \begin{equation} Li(x) = \int_2^x
\frac{dt}{\log t}, \end{equation} which estimates the number of
primes at most $x$) is italicized. To remove the italization we
write inside math mode \{ $\backslash$rm text not to be italicized
\}:
\begin{eqnarray}
\frac{1}{m}\sum_p^{m^\sigma} p^{-\foh} & \le & \frac{1}{m} \left(
\sum_p^{m^\sigma} \frac{1}{p}\right)^{\foh} \left(
\sum_p^{m^\sigma} 1\right)^{\foh} \nonumber\\ & \le & \frac{1}{m}
\left(\log \log m^{\sigma} + A\right)^{\foh} \left({\rm Li}(x) +
O\left(x^{\foh} \log x\right)\right)^{\foh} \nonumber\\ & \ll &
\frac{1}{m} (\log \log m)^{\foh} \left(\frac{2m^{\sigma}}{\log
m}\right)^{\foh} \nonumber\\ & \ll & m^{\foh \sigma - 1}
\left(\frac{\log \log m}{\log m}\right)^{\foh}.
\end{eqnarray}

Finally, instead of eqnarray one often uses align:
\begin{align} f(x,y) & = (x+y)^3 \nonumber\\ & = x^3 + 3x^2y + 3xy^2 + y^3.
\end{align} Note it spaces things slightly differently than
eqnarray:
\begin{eqnarray} f(x,y) & = & (x+y)^3 \nonumber\\ & = & x^3 + 3x^2y + 3xy^2 + y^3.
\end{eqnarray} If you want to use align and add additional spaces,
one can do $\backslash$space: \begin{align} f(x,y) &\ =\ (x+y)^3
\nonumber\\ &\ =\ x^3 + 3x^2y + 3xy^2 + y^3.
\end{align}

\subsection{Lemmas, Propositions, Theorems and Corollaries}

Now let's add a lemma. Below is how one would write it. Notice all
the English text is italicized. We'll follow the lemma immediately
with a proposition. With the way our file is set up, we start a
lemma with $\backslash$begin \{ lem \}, and end it with
$\backslash$end \{ lem \}.

\begin{lem} Let $\hat{\phi} (\xi) = \int_0^\infty \phi(x) e^{2 \pi i x} dx.$ Then
$\hat{\phi_r} (\xi) = \frac{1}{r} \hat{\phi}(\xi / r). $
\end{lem}

\begin{prop}
If $f \in \mathcal{C}^3$ and $f'(0) = 0, \ f''(0) > 0$ then $0$ is
a local minimum.
\end{prop}

\begin{proof} This follows immediately from the well known relation
\begin{equation}
3x + 2y = 4z,
\end{equation} which completes the proof. \end{proof}

Notice the proof environment above. We start with
$\backslash$begin \{ proof \}, follow it with the proof (which can
contain equations), and then end with $\backslash$end proof. The
nice thing is this automatically italicizes the word proof, and
ends with a box (which stands for QED, that which was to be shown,
the symbol often used to indicate the end of the proof). It is
important that the end proof be on the same line as the last text,
or there will be extra spacing.

\bigskip


\noindent \emph{Proof:} Here we do it without using the proof environment. This follows immediately from the well known relation
\begin{equation}
3x + 2y = 4z,
\end{equation} which completes the proof. \hfill $\Box$

\bigskip

The following is a new lemma, and what is inside the [] gives the
lemma a name.

\begin{lem}[Value of $\zeta(2)$]
Let $\zeta(s)$ denote the Riemann Zeta Function. Then
\begin{equation}
\zeta(2) = \frac{\pi^2}{6}.
\end{equation}
\end{lem}

You can label lemmas just like you would equations:

\begin{thm}\label{thm:implicitequationn}[The Implicit Equation] Let
$x, y, z \in \C$. Then
\begin{equation}
x^y + y^z + z^x = -1
\end{equation}
\end{thm}

If you view the file, you will notice that the name of the above
is italicized. It is better to put the name first, then the label:

\begin{thm}[The Implicit Equation]\label{thm:implicitequation} Let
$x, y, z \in \C$. Then
\begin{equation}
x^y + y^z + z^x = -1
\end{equation}
\end{thm}

By Theorem \ref{thm:implicitequation}, we see that the desired
expression equals $-1$. Note that you often have to compile twice
before the labels are correct.

\begin{cor}
$x = y = z = 1$ is not a solution to the Implicit Equation from
Theorem \ref{thm:implicitequationn}.
\end{cor}

\begin{cor}[Bob's Observation]
$x = y = -1, \ z = 1$ is a solution to the Implicit Equation from
Theorem \ref{thm:implicitequationn}.
\end{cor}

\begin{rek}[Seldon's Remark] Here is how you do a remark; notice
that the remark is italicized. \end{rek}

\subsection{Using Subsubsections}

If we wanted, we could put subsubsections in a section. Again,
$\backslash$section is section, $\backslash$subsection is a
subsection, $\backslash$subsubsection is a subsubsection; you then
put the name between braces.

\subsubsection{Pythagoras}

If a subsection is very long, we might want to have
sub-subsections in the subsections. The commands are exactly what
you think.

\begin{lem}[Lengths of Sides]
The sum of the lengths of any two sides of a triangle are greater
than the third length.
\end{lem}

\subsubsection{Garbage}

Here is another subsubsection.

Here is some more garbage text. \\ And here is some more.

Note the double $\backslash$above forces a carriage return. More
gobbledegook follows here, such as the quick brown fox and one
bright day in the middle of the night.

\noindent If you start a paragraph with $\backslash$noindent then
there is no indentation.



\subsubsection{New Pages}

We are now going to force a new page. The next subsubsection will
start on a new page. We do this by $\backslash$newpage.

\newpage


\subsubsection{Prime Number Theorem}

Below is the Prime Number Theorem. If we assume the Riemann
Hypothesis we can take $\alpha = \frac{1}{2} + \epsilon$:
\begin{equation}
\pi(x)\ = \ \frac{x}{\log x} \ + \ O(x^\alpha)
\end{equation}

In the next subsection we discuss tables. We add a slash newpage
command below to start it on a new page.

\newpage

\subsection{Tables}

We give an example of how to insert tables into a document. In
Tables \ref{tab:14families} and \ref{tab:2comparisons} we provide
some information from investigations of families of elliptic curves.
Note the structure of how we do a table. We start with a slash begin
\{table\} (and end with a slash end \{table\}). We then have the
center command (which is ended later). The next line is slash begin
\{tabular\}, followed by combinations of vertical lines and the
letters l, c, r. The letters stand for left, center and right, and
refer to how the text in that column is justified. The | give
vertical lines. Horizontal lines are inserted by the command slash
hline. Note the different entries are surrounded by ampersand signs
\&, and the rows are ended with slash slash. If you want a table
entry to be in math mode, simply surround it with dollar signs (see
Table \ref{tab:2comparisons}).

\textbf{It is very important that you put the table label in the
correct spot!}. After the slash end tabular, you may insert a
caption. This is done with the command: slash caption \{, followed
by all the captioning text you want, then ended with \}. Immediately
after the caption comes the label, which is of the form slash label
\{label name\}. \textbf{AFTER you have added the label, then you end
the centering, and then you end the table.}

\begin{table}
\begin{center}
\begin{tabular}{|l||c|c|c|c|r|}
  \hline
    \textbf{Family} &  \textbf{Median $\widetilde{\mu}$} &  \textbf{Mean $\mu$} &
    \textbf{StDev $\sigma_\mu$} &  \textbf{log(conductor)} &
    \textbf{Number}\\
    \hline \hline
\ \ 1: [0,1,1,1,T]   & 1.28  &  1.33  &  0.26 & [3.93, 9.66] &  7 \\
\ \ 2: [1,0,0,1,T]   & 1.39  &  1.40  &  0.29 & [4.66, 9.94] & 11 \\
\ \ 3: [1,0,0,2,T]   & 1.40  &  1.41  &  0.33 & [5.37, 9.97] & 11 \\
\ \ 4: [1,0,0,-1,T]  & 1.50  &  1.42  &  0.37 & [4.70, 9.98] & 20\\
\ \ 5: [1,0,0,-2,T]  & 1.40  &  1.48  &  0.32 & [4.95, 9.85] & 11\\
\ \ 6: [1,0,0,T,0]   & 1.35  &  1.37  &  0.30 & [4.74, 9.97] & 44\\
\ \ 7: [1,0,1,-2,T]  & 1.25  &  1.34  &  0.42 & [4.04, 9.46] & 10\\
\ \ 8: [1,0,2,1,T]   & 1.40  &  1.41  &  0.33 & [5.37, 9.97] & 11\\
\ \ 9: [1,0,-1,1,T]  & 1.39  &  1.32  &  0.25 & [7.45, 9.96] & 9\\
10: [1,0,-2,1,T]  & 1.34 &   1.34 &   0.42 & [3.26, 9.56] & 9\\
11: [1,1,-2,1,T]  & 1.21 &   1.19 &   0.41 & [5.73, 9.92] & 6\\
12: [1,1,-3,1,T]  & 1.32 &   1.32 &   0.32 & [5.04, 9.98] & 11\\
13: [1,-2,0,T,0]  & 1.31 &   1.29 &   0.37 & [4.73, 9.91] & 39\\
14: [-1,1,-3,1,T] & 1.45 &   1.45 &   0.31 & [5.76, 9.92] & 10\\
\hline\hline
    \textbf{All Curves}  & 1.35 &   1.36 &   0.33  &  [3.26, 9.98] &
    209\\
    \textbf{Distinct Curves} & 1.35 & 1.36 & 0.33 & [3.26, 9.98] &
    196
    \\
  \hline
\end{tabular}
\caption{First normalized zero above the central point for $14$
one-parameter families of elliptic curves of rank $0$ over $\Q$
(smaller conductors)} \label{tab:14families}
\end{center}
\end{table}

Note the tables do not always appear where you want them to. You can
try to force them to be in certain places, but the TeX program puts
the tables around where you want, taking into account how much space
is available on the pages. Note, for instance, that only this
paragraph is between the two tables, but they appear on different
pages.



\begin{table}
\begin{center}
\begin{tabular}{|l||c|c|c|c|}
  \hline
\textbf{Family} & \textbf{Median} & \textbf{Mean} & \textbf{Std.
Dev.} & \textbf{Number} \\ \hline
Rank 2 Curves, Families Rank 0 over $\Q$ & 1.926 & 1.936 & 0.388 & 701 \\
Rank 2 Curves, Families Rank 2 over $\Q$ & 1.642 & 1.610 & 0.247 & \ \ 64 \\
\hline
\end{tabular}
\caption{First normalized zero above the central point. The first
family is the 701 rank $2$ curves from the $21$ one-parameter
families of rank $0$ over $\Q$ from Table 3 with $\log({\rm cond})
\in [15,16]$; the second family is the 64 rank $2$ curves from the
$21$ one-parameter families of rank $2$ over $\Q$ from Table $4$
with $\log({\rm cond}) \in [15,16]$.} \label{tab:2comparisons}
\end{center}
\end{table}




\subsection{Matrices and Shortcuts}

If you have symbols you use many times, it is often convenient to
define a shortcut. For example, I have defined $\backslash$foh to
be $\frac{1}{2}$ (stands for fraction: one half).

Let's do some more detailed examples. To do a $5 \times 5$ matrix
type
\begin{equation}
A =  \left( \begin{array}{ccccc}
                        1  & 2  & 3  & 4  & 5  \\
                        6  & 7  & 8  & 9  & 10 \\
                        11 & 12 & 13 & 14 & 15 \\
                        16 & 17 & 18 & 19 & 20 \\
                        16 & 17 & 188 & 19 & 20 \\
                        121 & 122 & 123 & 124 & 125
                          \end{array}\right)
\end{equation}

Some important points to note: the $\backslash$left and the
$\backslash$right draw the left and right parentheses around the
matrix, automatically adjusting to the proper size. Then we have a
$\backslash$begin array, followed by five cs in curly braces. This
gives a $5 \times 5$ matrix, with each element centered. If
instead of cs we used ls, it would left-justify (and rs would
right justify). For example,
\begin{equation}
A =  \left( \begin{array}{rrrrr}
                        1  & 2  & 3  & 4  & 5  \\
                        6  & 7  & 8  & 9  & 10 \\
                        11 & 12 & 13 & 14 & 15 \\
                        16 & 17 & 18 & 19 & 20 \\
                        121 & 122 & 123 & 124 & 125
                          \end{array}\right)
\end{equation}

Notice we use ampersand signs \& between the various elements;
this is similar to multi-line equations, where we used the
ampersands to center things. We end each line with
$\backslash$$\backslash$, which is a carriage return.

Latex allows you to define shortcuts that are functions of up to
nine arguments with ease. Thus, I can have shortcut definitions
for $2 \times 2$ and $3 \times 3$ matrices. To use my shortcuts
(the newcommands at the top), one just types
\begin{equation}
A = \mattwo{a}{b}{c}{d}
\end{equation}
or
\begin{equation}
B = \matthree{a}{b}{c}{d}{e}{f}{g}{h}{i}.
\end{equation}

If you want determinants, you can do
\begin{equation}
\det(A) = \dettwo{a}{b}{c}{d}, \ \ \ \det(B) =
\detthree{a}{b}{c}{d}{e}{f}{g}{h}{i}.
\end{equation}

Two remarks on the above equation. First, $\backslash$det gives a
nice determinant. For functions like det, sin, cos, if you don't
put a $\backslash$before them, Latex interprets them as text
(letters). Thus, compare $\det(A), \cos(A), \sin(A), \log(A)$ to
$det(A), cos(A), sin(A), log(A)$.

Second, a $\backslash$followed by a space gives a space. Latex
ignores (for the most part) spaces. In the above, the
$\backslash$space $\backslash$space $\backslash$space gives three
spaces (ie, helps format).

Finally, here are some other shortcuts I've created that you might
find useful. Feel free to make your own!

Here is one to do a two case argument:
\begin{equation}
\twocase{\Lambda(n) = }{\log p}{if $p$ is a prime
power}{0}{otherwise.}
\end{equation}

Here is a way to do a three case argument.
\begin{equation}
\threecase{\mu(n) = }{1}{if $n = 1$}{(-1)^r}{if $n$ is the product
of $r$ distinct primes}{0}{if $n$ is divisible by the square of a
prime.}
\end{equation}

For the Legendre / Jacobi symbol with modulus $p$, simple use
$\js{x}$.

The standard sets of numbers encountered are the natural numbers
$\N$, the integers $\Z$, the reals $\R$, and the complex numbers
$\C$.

If we want to do vectors, we just do $\overrightarrow{v}$; set
operations are $x \in A \cup B \cap C \subset D = G \oplus H$.

Sometimes we prefer to write \be x \ = \ \bigcup_{i=1}^\infty A_i
\ \ \ instead of \ \ \ x \in \cup_{i=1}^\infty A_i. \ee

Note the above has the text in emphasis mode. To avoid this, use
either mbox or text (text is better): \be x \ = \
\bigcup_{i=1}^\infty A_i \ \ \ \mbox{instead of} \ \ \ x \in
\cup_{i=1}^\infty A_i. \ee

Here it is with text instead of mbox. \be x \ = \
\bigcup_{i=1}^\infty A_i \ \ \ \text{instead of} \ \ \ x \in
\cup_{i=1}^\infty A_i. \ee

Here it is using remove emphasis $\backslash$rm: \be x \ = \
\bigcup_{i=1}^\infty A_i \ \ \ {\rm instead of} \ \ \ x \in
\cup_{i=1}^\infty A_i. \ee Note with the remove emphasis that we
have lost the space between instead and of; to add it back we
would need a $\backslash$space.

To do a unit vector, one can write $\widehat{i}, \widehat{j},
\widehat{k}$. We also have $\prod_{i=1}^5 \sum_{j=1}^8 a_{ij}$.

We have now come to the end of the first section -- it will
automatically start the next section on a new page.

I like to put a lot of percent signs between sections (and a few
carriage returns) to make editing easier.


%%%%%%%%%%%%%%%%%%%%%%%%%%%%%%%%%%%%%%%%%%%%%%%%%%%%%%%%%%%%%%%%%
%%%%%%%%%%%%%%%%%%%%%%%%%%%%%%%%%%%%%%%%%%%%%%%%%%%%%%%%%%%%%%%%%

\section{Environments I}

\subsection{Shortcut Environments}

To do an equation, recall we need $\backslash$begin curly brackets
equation curly brackets. Thus, we write

\begin{equation}
\sum_{n=1}^\infty \frac{1}{n^2} = \frac{\pi^2}{6} = \frac{8}{3}
\left(\int_0^1 \frac{1}{1+x^2}dx\right)^2. \end{equation}

If you have a lot of equations or arrays of equations, you don't
want to keep typing begin equation and end equation.

We've created some shortcuts: $\backslash$be will be begin
equation; $\backslash$ee will end the equation; bea and eea will
begin and end arrays of equations.

Thus, \be\sum_{n=1}^\infty \frac{1}{n^2} = \frac{\pi^2}{6} =
\frac{8}{3} \left(\int_0^1 \frac{1}{1+x^2}dx\right)^2  \ee does it
as an equation, and \bea \sum_{n=1}^\infty \frac{1}{n^2} &=&
\frac{\pi^2}{6} \nonumber\\ &=& \frac{8}{3} \left(\int_0^1
\frac{1}{1+x^2}dx\right)^2 \eea

does it as an array of equations.

Other useful commands: $\backslash$textbf \} text you want in bold
\{: \textbf{this will put any text in bold}; $\backslash$emph \{
text you want emphasized \}: \emph{this will emphasize or
italicize text}; $\backslash$underline \{ text to underline \}:
\underline{this will underline}.


\subsection{Enumeration, Itemizing, and General Latex and Linux Commands}

We will use the shortcuts for the enumeration environment. First,
the long form. We start with $\backslash$begin \{ enumerate \}.
This starts the enumeration (ie, a list of items with each item
numbered). Each item in the enumerated list starts with
$\backslash$item. To have sub-items we use $\backslash$subitem;
however, there are no bullet points or marks here; we can add a
bullet with $\bullet$. We end with $\backslash$end \{ enumerate
\}.

\begin{enumerate}

\item in whatever directory you want to latex, save the files
template.tex, yl.eps. \subitem This will give you a tex template
with the image yl.eps. \subitem $\bullet$ Not the \$
$\backslash$bullet \$ gives a bullet point.

\item The following commands work in many unix environments: at
the unix prompt, move into the directory where you've saved the
templates. To edit, type emacs template.tex \& (the ampersand
makes sure it opens in a new window). To compile is
\textbf{Control-c-f}. Type xdvi template.dvi \& to view your
compiled file. If you make changes to the latex source file, just
clicking on the xdvi file will (if you've compiled the latex file)
automatically update the dvi file. control-c (let go of the two
keys) ` displays error messages (it's on the same key as the
tilde, don't hold down shift).

\end{enumerate}

If instead of numbering we wanted bullets, we just change
enumerate to itemize. Here $\backslash$subitem also does not give
bullet points. Instead of using bullets we could use diamonds,
$\backslash$diamond.

\begin{itemize}

\item To open a file for editing using emacs, type emacs filename
\&; the ampersand opens the file in a new window. \subitem
$\clubsuit$ Control-x-w saves without exiting; \subitem $\diamond$
Control-x-c saves with exiting.

\item General: We list some general commands: \subitem $\diamond$
cd directory changes the directory, \subitem $\diamond$ ls lists
all files and sub-directories in current directory.

\end{itemize}

You can also use my shortcuts for these environments.

\ben

\item ben stands for begin enumerate.

\item een stands for end enumerate.

\item each line starts with $\backslash$item.

\een

Itemize is similar.

\bi

\item bi is begin itemize.

\item ei is end itemize.

\item each line starts with $\backslash$item.

\ei


%%%%%%%%%%%%%%%%%%%%%%%%%%%%%%%%%%%%%%%%%%%%%%%%%%%%%%%%%%%%%%%%%
%%%%%%%%%%%%%%%%%%%%%%%%%%%%%%%%%%%%%%%%%%%%%%%%%%%%%%%%%%%%%%%%%

\section{Graphics and Color}

\subsection{Inserting Graphics}

Let's end by inserting a picture (image courtesy of J. Ax and S.
Kochen). The image extension should be .eps, and in the same
directory as everything. We display it in Figure \ref{fig:yl}

\begin{figure}
\begin{center}
\scalebox{.5}{\includegraphics{yl.eps}}
\caption{\label{fig:yl} The infamously famous yl image.}
\end{center}
\end{figure}

On some systems, if I try to view the .dvi file I have trouble
seeing the picture; I need to convert it to a .ps file and then
use ghostview or some such.

\subsection{Color}

If you have included the color package, you can write in color.
Simply use the following:

{\rmfamily\color[rgb]{.55,0,.15}This is to type in a shade of red.
There are three parameters, red green blue, I think. Each is a
number between 0 and 1. Note you start with a brace followed by
rmfamily then the color specification, and you end with another
brace like this.}

If we type here, it is in black, the default color.

{\color[rgb]{0,0,.85} Now we are in a shade of blue. You can also
use \textbf{bold test} in a color, as well as math notation, such
as $a_i^j x_{ij} = \pi^e$, or whatever you want.}

We can also {\color[rgb]{1,0,0}switch colors to pure red} or to
{\color[rgb]{0,0,1}pure blue} or back to black in the middle of a
paragraph.

%%%%%%%%%%%%%%%%%%%%%%%%%%%%%%%%%%%%%%%%%%%%%%%%%%%%%%%%%%%%%%%%%
%%%%%%%%%%%%%%%%%%%%%%%%%%%%%%%%%%%%%%%%%%%%%%%%%%%%%%%%%%%%%%%%%

\section{Environments II}

\subsection{Lists}

Enumerating lists:

\begin{enumerate}

\item This is the first item.
\item This is the second item.
\item This is the last item.

\end{enumerate}


\subsection{Emphasize and Bolding}

If you use \textbf{this, then whatever is inside will be in bold},
while if you use \emph{this, everything will be emphasized}, and
\underline{this will cause the text to be underlined}.

I have created a shortcut for textbf, namely \tbf{this will bold
text as well}.


\subsection{Centering Text}

One can also center text:

\begin{center}

Everything typed in here is centered.

Isn't centering wonderful?

I thought so too.

\end{center}


This is also wonderful. \\

The two slashes above give a extra carriage return. You can only
have one double slash at the end of a line. If you want more, use
bigskip.


\subsection{Refering to Bibliography}

The bibliography is included at the end. To refer to items, simply
type \cite{RSZ}. Note all the items in the bibliography have two
abbreviations, one in brackets (which is what is displayed in the
bibliography), and one in curly braces (the shortcut name); I
often have the two the same..

Again, what is in brackets is what the computer will print; what
is in curly braces is how you refer to it.

Thus, to refer to the book by Khinchin you should type
$\backslash$Kh: \cite{Kh}. We can also refer to the appendices,
such as see Appendix \ref{sec:sayshell} or see
\S\ref{sec:sayshell}.

\subsection{Font sizes}

We describe some different font sizes:

\begin{itemize}

\item \huge This is huge.

\item \LARGE This is LARGE

\item \Large This is Large

\item \normalsize You guessed it.

\item \tiny This is tiny.

\item \normalsize Back to normalsize.

\end{itemize}

Notice in the above how the bullet size changes; the reason is
that we start off in normal size, but when we type
$\backslash$huge we enter huge mode; we are in huge mode until we
type $\backslash$LARGE, when we enter LARGE mode. Thus the bullet
points are adjusted.

Consider the matrix $A = \mattwo{a}{b}{c}{d}$. Boy does this look
bad compiled. How about $A = \tiny \mattwo{a}{b}{c}{d}$. This
looks better.


\section{Document class and global formatting}

First off, you can adjust the margins of your document. Below are
the commands used in this document -- by playing with the numbers
you can obtain the margins you want. We have used the verbatim
command (slash begin \{verbatim\}) so that the TeX program will not
convert our commands (ie, so it will just display them).

\begin{verbatim}
\addtolength{\textwidth}{2cm} \addtolength{\hoffset}{-1cm}
\addtolength{\marginparwidth}{-1cm} \addtolength{\textheight}{2cm}
\addtolength{\voffset}{-1cm}

\end{verbatim}

You can also make the document double spaced easily. For example,
type
\begin{verbatim}\baselinestretch}{2}\end{verbatim} at the beginning
of the document and your document will be double spaced. If you use
a number between 1 and 2, you get something between single and
double spaced.

We have used the document class amsart (AMS article) and a font size
of 12: this is evidenced by the command (at the start of the file)
\begin{verbatim}\documentclass[12pt,reqno]{amsart}\end{verbatim}
The reqno means right equation numbers.

We have also used (in the beginning) \begin{verbatim}
\subjclass[2000]{ (primary),  (secondary).}

\keywords{How to use TeX}
\end{verbatim}

This inserts the subject classification numbers and keywords. The
subject classifications can be found online at
$$\texttt{http://www.ams.org/msc/}.$$

Finally, we remark that an article class of amsart is not the only
choice: one can also use article, report, book, .... More on these
can be found online.

\section{Further Reading}

There are numerous sources online for additional TeX help. See for
example $$\texttt{http://www.giss.nasa.gov/tools/latex/}$$ or
$$\texttt{http://www.stat.washington.edu/software/latex/}$$ or all
the sites referenced at
$$\texttt{http://coulomb.ecn.purdue.edu/~bulsara/LaTeX/latex.html}$$

\newpage

\section*{Acknowledgements}

Many of these shortcut commands are from a .tex template that Alex
Barnett was kind enough to share with me.

%%%%%%%%%%%%%%%%%%%%%%%%%%%%%%%%%%%%%%%%%%%%%%%%%%%%%%%%%%%%%%%%%
%%%%%%%%%%%%%%%%%%%%%%%%%%%%%%%%%%%%%%%%%%%%%%%%%%%%%%%%%%%%%%%%%

\appendix

\section{Psychohistorical Dynamics of the Sayshell Republic: An Analysis of
the Rise of the Mule}\label{sec:sayshell}

This is the first appendix, works like you would expect.

\section{Random Walks in High Dimensions: Choosing a Universe Interesting
for Drunks}

This is the second appendix.







%%%%%%%%%%%%%%%%%%%%%%%%%%%%%%%%%%%%%%%%%%%%%%%%%%%%%%%%%%%%%%%%%
%%%%%%%%%%%%%%%%%%%%%%%%%%%%%%%%%%%%%%%%%%%%%%%%%%%%%%%%%%%%%%%%%

% note the form of the bibliography. start with a bibitem.
% you refer to an entry in the bibliography by using the label in braces
% the computer prints the label in brackets. often I have both the same, but
% you don't have to.

\begin{thebibliography}{9999}

%number of characters controls spacing; must be longer than longest name

\bibitem[Da]{Da}
\newblock H. Davenport, \emph{Multiplicative Number Theory}, second
edition, Graduate Texts in Mathematics \textbf{74},
Springer-Verlag, New York, 1980, revised by H. Montgomery.

\bibitem[Kh]{Kh}
\newblock A. Y. Khinchin, \emph{Continued Fractions},  Third
Edition, The University of Chicago Press, Chicago 1964.

\bibitem[RSZ]{RSZ}
\newblock Z. Rudnick, P. Sarnak, and A. Zaharescu, \emph{The
Distribution of Spacings Between the Fractional Parts of
$n^2\alpha$}, Invent. Math. \textbf{145} (2001), no. 1, 37--57.

\end{thebibliography}

\bigskip

\end{document}
